\documentclass[11pt, a4paper]{scrartcl}

\usepackage{fullpage}

\newcommand{\email}[1]{\small{#1}}
\newcommand{\cname}[1]{\mbox{\texttt{#1}}}

\title{Advanced Functional Programming}
\subtitle{Tic-tac-logic}
\author{Christos Sakalis \\ \email{Christos.Sakalis.3822@student.uu.se}
    \and Aliaksandr Ivanou \\ \email{Aliaksandr.Ivanou.1364@student.uu.se}}

\begin{document}

\maketitle

\section{Introduction}

The tic-tac-logic application was worked on mostly by Christos (me). Alex helped
in the planning stage by having someone to bounce some ideas off and during the
coding stage by code reviews.

There are three Haskell files:

\begin{enumerate}
    \item \cname{tic-tac-logic.hs}: The file that contains the main function and
        is responsible for IO.
    \item \cname{Game.hs}: The Game module contains all the data types and
        functions used for representing, accessing and solving the puzzle.
    \item \cname{Test.hs}: The Test module contains all the tests performed on
        the Game module.
\end{enumerate}

There is also a Makefile to make compiling the main executable and, if required,
the test executable as well.

\section{Implementation}

\subsection{The Board}

The game board is a square grid. Is is represented by a Map with a pair of Int
as the key. The first member represents the row and the second the column. The
reason we use a map is because it allows for both quick random access and quick
modifications to the board. Haskell's arrays on the other hand are much slower
when they need to be altered.

The board also keeps a \emph{isTransposed} value. The reason we have this value is
because we do not want to have to duplicate our functions that work on rows as
to make them to work on columns too. Transposing the board is very cheap as it
only changes the transposed value and exchanges the number of rows and columns.
The actual board remains the same. The functions that are responsible for
reading and writing marks from the board take care to read or write to the
correct place, depending if the board is transposed or not. No other functions
care for this value. This means that transposing the board has practically zero
cost, while at the same time saving a lot of code and added complexity.

Some of the operations we need to do in order to solve the puzzle work on rows
or columns of the board. For that reason we also have a Row type that allows us
to extract and work on a row on its own. Because we can transpose the board we
do not need a type for the columns.

\subsection{The Logic}

There are four simple methods for solving such puzzles:

\begin{enumerate}
    \item Pairs of the same mark have to be surrounded by marks of the other
        type
    \item When two marks of the same type are separated by a blank, the blank
        has to be filled with a mark of the other type
    \item When the number of marks of one type are half the total number of
        spaces in the row, then the remaining spaces have to be filled with the
        other type
    \item Checking if filling one row would cause row duplication in the board
\end{enumerate}

With the exception of the 4th method, all the other methods can be used in a row
without knowledge of the rest of the board. Also, the first three methods do not
require any speculation on the solver's part.

The first two methods are essentially part of the same rule (no more that two
marks of the same type next to each other) and we apply them using the same
function. All we do is get a row and check to see if we have any triplets that
match any of the two criteria. If we do, then we complete them.

The third method is equally simple. We get a row, we count the number of Xs and
Os and if one of them is complete them we fill the remaining spaces with the
other one. This method comes from the rule that each row should contain an equal
number of Xs and Os.

The fourth method is not that simple because row duplication is not immediately
obvious. Occasions where we have a row completed and another matching row with
just two spaces left are rare. For that reason we do not explicitly use this
method in our solver, but it is included in the speculation part.

By speculation we mean trying to find a correct mark by guessing. We do not
however do exhaustive search. Instead, we pick an empty spot on the grid and
insert an X and O there. Then we try and complete the board with \textbf{the
three other} methods. Finally, we check if the resulting board are valid. If one
of the boards is not valid, then that means that the other board is the only
valid choice available, and so we make that choice and put the mark in that
position.

It is obvious how speculation covers the row duplication method. As a matter of
fact, it covers all the other methods as well, but we prefer to use the other
methods first because they are much faster that just getting all the empty
places on the grid and trying random marks on them. In the case of the row
duplication however, the cases where a separate function would have worked are
very few and so it is not worth the extra code and complexity added.

In order to solve the puzzle, we apply the first three methods until there is
nothing more that we they are able to do. Then, we proceed by using speculation
on the whole board. After speculation is done, then we again use the first three
methods etc. until no more changes can be made by any of the methods. At the
point we stop, because either we have completed the puzzle of there is not a
unique or valid solution to it.

\section{Testing}

For testing the implemented functions automated and manual testing was used.
Specifically, QuickCheck was used for the automated testing and HUnit for
manual. The reason for using HUnit and not QuickCheck for all the functions is
that the many of the functions that are part of the solving logic have very
specific input requirements. Generating random inputs and validating those
requirements would be too hard to do. So some basic unit tests were generated to
test that the functions are working properly.

By running the tests with HPC we verified that they cover all the code we wanted
to check. For example, we do not care about the printing functions.

All the tests can be run by first compiling the \cname{Test} executable through
\cname{make test} and then running it. The execute is not compiled automatically
with the \cname{tic-tac-logic} one because it requires some extra Haskell
libraries to be installed. Some language extensions are used as well.

\section{Extras}

Since there were no extra requirements, here are some awesome Haskell features
that were used in the project. They were mostly used to make the code shorted,
facilitate code reuse and thus decrease the number of mistakes made.

\begin{description}
    \item[Records] Records are nothing but tuples in Haskell, but allow named
        access of their members. More importantly, they allow for the creation
        of new records based on existing ones by altering only the members we
        want and having the rest being copied automatically.
    \item[Instances] The only derived instance used is for having the boards
        being compared automatically. We have implemented our own instances for
        comparing and printing rows as well as printing boards. In the testing
        module, we use instances of Arbitrary for generating random boards.
    \item[Extensions] There is a number of language extensions supported by GHC.
        We use them in testing for two reasons. First of all, it allows us to
        declare instances of type aliases, which is not normally allowed in
        Haskell. Second, running all the prop\_ QuickCheck tests automatically
        requires the template Haskell metaprogramming extension.
    \item[Flexible Syntax] Function composition, the low precedence bind
        operator and partial bound functions allow the code to be much more
        compact.
    \item[Laziness] Many of the functions are much simpler because of laziness.
        For example, isRowValid liberally uses list manipulation functions
        that affect the whole list even though we want to stop after the first
        mismatch. Because of laziness these functions might never access the
        whole list.
    \item[Type System] All the functions have type annotations. This not only
        helps the programmer make less mistakes but also works as partial
        documentation as to what the function does.
    \item[HPC] Haskell Program Coverage assures us that our test cases
        cover the functions we wanted to check.
\end{description}

\end{document}

who did what
description
test
extras
